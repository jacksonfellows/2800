\documentclass{article}

\usepackage{amsthm}
\usepackage{amsmath}

\providecommand{\implies}{}
\renewcommand{\implies}{\Rightarrow}
\newcommand{\twowayimplies}{\Leftrightarrow}

\theoremstyle{remark}
\newtheorem{case}{Case}
\newtheorem{subcase}{Case}[case]

\gdef\author{Jackson Fellows --- jf787}

\newcommand{\class}[2]{
  \gdef\classnumber{#1}
  \gdef\classname{#2}
}
\newcommand{\homework}[2]{\gdef\title{Homework #1: #2}}
\newcommand{\due}[1]{\gdef\duedate{#1}}

\renewcommand{\maketitle}{
  \setlength\parindent{0in}
  \addtolength\parskip{1ex}
  \setlength\fboxrule{.5mm}\setlength{\fboxsep}{1.2mm}
  \fbox{
    \parbox{\textwidth}{
      \begin{center}
        \large\textbf{\title}\\
        \ \\
        \classnumber \hfill \author\\
        \classname \hfill Due \duedate\\
      \end{center}
    }
  }
  \bigskip
}

\begin{document}

\class{CS2800}{Discrete Structures}
\homework{1}{Proofs and Logic}
\due{February 2}
\maketitle

\begin{enumerate}
\item
  The proof given does not prove what is intended by the question (that the arithmetic mean of two numbers is at least as large as their geometric mean).
  Instead, it simply shows that starting with the proposition that the arithmetic mean is at least as large as the geometric mean, we can arrive at the trivially true proposition $(a - b)^2 > 0$.
  That is, it is a proof that $\frac{a + b}{2} \ge \sqrt{ab} \implies (a - b)^2 > 0$, which is not very useful.
  To actually prove the statement in question, we need to start with the true proposition and end with what we want to prove.
  \begin{proof}
    Let $a$ and $b$ be non-negative real numbers.
    Then
    \begin{align*}
      (a - b)^2 &\ge 0, &\mbox{which implies that}\\
      a^2 - 2ab + b^2 &\ge 0, &\mbox{which implies that}\\
      a^2 + 2ab + b^2 &\ge 4ab, &\mbox{which implies that}\\
      a + b &\ge 2\sqrt{ab}, &\mbox{which implies that}\\
      \frac{a + b}{2} &\ge \sqrt{ab}.
    \end{align*}
  \end{proof}
  Since we started with a self-evidently true statement, this proves that the arithmetic mean is at least as large as the geometric mean.
  Note that the step where we took the square root of each side of the inequality relies on the stated assumption that $a$ and $b$ are non-negative, since this means that both sides of the inequality are guaranteed to be positive.
\item
  Let $a$ and $b$ be real numbers.
  \begin{enumerate}
  \item
    We want to prove that if $n = \frac{a + b}{2}$ then $a$ or $b$ are at most $n$.
    \begin{proof}
      We will prove the contrapositive that if $a$ and $b$ are greater than $n$ then $n \ne \frac{a + b}{2}$.
      Given $a > n$ and $b > n$, it follows that $a + b > 2n$.
      From the above, it follows that $\frac{a + b}{2} > n$.
      $n$ cannot equal something greater than itself, so $n \ne \frac{a + b}{2}$.
    \end{proof}
  \item
    We want to prove that $\left|a + b\right| \leq \left|a\right| + \left|b\right|$.
    \begin{proof}
      We will use a proof by cases.
      \begin{case}[Both $a$ and $b$ are positive]
        If $a$ and $b$ are positive, then $\left|a + b\right| = a + b$, $\left|a\right| = a$, and $\left|b\right| = b$.
        Therefore, the conditional in question reduces to $a + b \leq a + b$ which is true since both sides are equal.
      \end{case}
      \begin{case}[Either $a$ or $b$ is negative]
        Let's say that $a$ is negative and $b$ is positive.
        Then we can assume that $a = -k$ where $k$ is a positive number.
        Now, $\left|a\right| = k$ and $\left|b\right| = b$.
        To simplify $\left|a + b\right|$ we need a new case for whether or not $b$ is greater than or equal to $k$.
        \begin{subcase}[$b$ is greater than or equal to $k$]
          In this case, $\left|a + b\right| = \left|b - k\right| = b - k$.
          Now we have $b - k \leq k + b$.
          This simplifies to $-k \leq k$ which is always true for positive $k$.
        \end{subcase}
        \begin{subcase}[$b$ is less than $k$]
          In this case, $\left|a + b\right| = \left|b - k\right| = k - b$.
          Now we have $k - b \leq k + b$.
          This simplifies to $-b \leq b$ which is always true for positive $b$.
        \end{subcase}
      \end{case}
      \begin{case}[Both $a$ and $b$ are negative]
        We can assume that $a = -k$ and $b = -l$ where $k$ and $l$ are positive numbers.
        Now, $\left|a + b\right| = \left|-k - l\right| = k + l$, $\left|a\right| = k$, and $\left|b\right| = l$.
        The conditional now simplifies to $k + l \leq k + l$ which is true since both sides are equal.
      \end{case}
    \end{proof}
  \end{enumerate}
\item
  \begin{enumerate}
  \item
    We want to prove that every odd integer can be written as the difference of two perfect squares.
    \begin{proof}
      Consider two consecutive perfect squares $n^2$ and $(n+1)^2$ for an integer $n$.
      Their difference is $(n+1)^2 - n^2 = n^2 + 2n + 1 - n^2 = 2n + 1$.
      Since every odd number can be written as $2k + 1$ for an integer $k$, every odd number can be written as the difference of two perfect squares.
    \end{proof}
  \item
    We want to prove that there exist even integers that can be written as the difference of two perfect squares, and that there also exist even integers that cannot.
    \begin{proof}
      Consider the two arbitrary perfect squares $n^2$ and $(n + j)^2$ for the integers $n$ and $j$.
      Their difference is $(n + j)^2 - n^2 = n^2 + 2nj + j^2 - n^2 = 2nj + j^2 = j(2n + j)$.
      For this expression to be even $j$ must be even.
      Then $j = 2k$.
      Now, $j(2n + j) = 2k(2n + 2k) = 4kn + 4k^2 = 4k(m + 1)$.
      Since $k$ and $m$ are integers, this means that the only even differences of perfect squares are multiples of 4.
      Therefore, some even integers (multiples of 4) can be written as the difference of perfect squares, while some (those that are not multiples of 4) cannot.
    \end{proof}
  \item
    \begin{enumerate}
    \item
      $\exists n.\ E(n) \land D(n)$.
      This statement is true.
      \begin{proof}
        Take $n = 8$.
        $3^2 - 1^2 = 9 - 1 = 8$.
        Therefore there exists an $n$ which is even and a difference of two perfect squares.
      \end{proof}
    \item
      $\exists n.\ E(n) \land \neg D(n)$.
      This statement is true.
      \begin{proof}
        In part b we proved that the only even integers that are the difference of perfect squares are multiples of 4.
        Therefore, if we take $n = 2$ we have an even number which is also not the difference of two perfect squares.
      \end{proof}
    \item
      $\forall n.\ E(n) \implies \neg D(n)$.
      This statement is false.
      \begin{proof}
        We have already shown above that there exists an $n$ which is both even and a difference of two perfect squares.
        Therefore, it cannot be true that for all even integers $n$ that $n$ is not the difference of two perfect squares.
      \end{proof}
    \item
      $\forall n.\ \neg E(n) \implies D(n)$.
      This statement is true.
      \begin{proof}
        In part a we proved that every odd integer could be written as the difference of two perfect squares.
        Therefore, the fact that $n$ is odd does imply that $n$ is the difference of two perfect squares.
      \end{proof}
    \item
      $\forall n.\ \neg E(n) \twowayimplies D(n)$.
      This statement is false.
      \begin{proof}
        To prove this statement we must prove both that $\neg E(n) \implies D(n)$ (which we just proved above) and that $D(n) \implies \neg E(n)$.
        This second implication cannot be true because as we have proven above there do exist even integers which are differences of two perfect squares.
      \end{proof}
    \end{enumerate}
  \end{enumerate}
\item
  \begin{enumerate}
  \item
    \begin{displaymath}
      \begin{array}{|c|c|c|c|c|c|}
        \hline
        Q & P & P \implies Q & Q \land (P \implies Q) & (Q \land (P \implies Q)) \implies P\\
        \hline
        T & T & T & T & T\\
        T & F & T & T & F\\
        F & T & F & F & T\\
        F & F & T & F & T\\
        \hline
      \end{array}
    \end{displaymath}
    This is not a tautology.
  \item
    \begin{displaymath}
      \begin{array}{|c|c|c|c|c|c|c|c|}
        \hline
        Q & P & \neg Q & P \implies Q & \neg Q \land (P \implies Q) & \neg P & (\neg Q \land (P \implies Q)) \implies \neg P\\
        \hline
        T & T & F & T & F & F & T\\
        T & F & F & T & F & T & T\\
        F & T & T & F & F & F & T\\
        F & F & T & T & T & T & T\\
        \hline
      \end{array}
    \end{displaymath}
    This is a tautology.
  \item
    \begin{displaymath}
      \makeatletter
      \hspace*{-6.25cm}         % bad
      \makeatother
      \begin{array}{|c|c|c|c|c|c|c|c|c|c|}
        \hline
        P & Q & R & P \implies Q & Q \implies R & R \implies P & (P \implies Q) \land (Q \implies R) \land (R \implies P) & P \land Q \land R & ((P \implies Q) \land (Q \implies R) \land (R \implies P)) \implies (P \land Q \land R)\\
        \hline
        T & T & T & T & T & T & T & T & T\\
        T & T & F & T & F & T & F & F & T\\
        T & F & T & F & T & T & F & F & T\\
        T & F & F & F & T & T & F & F & T\\
        F & T & T & T & T & F & F & F & T\\
        F & T & F & T & F & T & F & F & T\\
        F & F & T & T & T & F & F & F & T\\
        F & F & F & T & T & T & T & F & F\\
        \hline
      \end{array}
    \end{displaymath}
    This is not a tautology.
  \end{enumerate}
\end{enumerate}
\end{document}
